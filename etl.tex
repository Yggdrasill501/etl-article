\documentclass{article}
\usepackage{graphicx}
\usepackage{hyperref}

\title{ETL/ELT Pipelines: Komplexný sprievodca}
\author{}
\date{}

\begin{document}

\maketitle

\section{Úvod}

Zabezpečenie bezproblémovej integrácie dát medzi rôznymi platformami je pre každú organizáciu kľúčové. Procesy transformácie dát sú nevyhnutné na spracovanie rôznorodých dátových súborov a moderné nástroje a techniky umožňujú efektívnejšiu správu týchto transformácií. ETL (Extract, Transform, Load) a ELT (Extract, Load, Transform) sú zásadné pre oblasť dátového inžinierstva, poskytujú rámce na presun, spracovanie a transformáciu dát.

V tomto článku sa pozrieme na ETL a ELT pipelines, ich význam a ako nastaviť tieto procesy na zjednodušenie integrácie a správy dát.

\section{Pochopenie ETL pipelines}

ETL pipelines predstavujú tradičnú metódu pre integráciu dát, kde sa dáta najprv extrahujú z rôznych zdrojových systémov, potom sa transformujú do vhodného formátu a nakoniec sa načítajú do cieľového systému, ako je dátový sklad alebo dátové jazero. ETL je obzvlášť výhodné, keď dáta potrebujú byť vyčistené, obohatené alebo preštruktúrované pred ich uložením alebo analýzou.

Tri hlavné fázy ETL sú:

\begin{itemize}
    \item \textbf{Extract:} Extrakcia dát zo zdrojových systémov, ktoré môžu zahŕňať databázy, API, súbory alebo iné zdroje dát.
    \item \textbf{Transform:} Transformácia dát, ako je filtrovanie, agregácia alebo konverzia formátov dát na splnenie obchodných požiadaviek.
    \item \textbf{Load:} Uloženie transformovaných dát do cieľového systému, zvyčajne do dátového skladu, dátového jazera alebo inej analytickej platformy.
\end{itemize}

\section{Výhody ETL pipelines}

ETL pipelines sú obzvlášť účinné, keď dáta vyžadujú komplexné transformácie alebo predspracovanie pred uložením. Hlavné výhody ETL pipelines sú:

\begin{itemize}
    \item \textbf{Kvalita dát:} Transformácia môže zabezpečiť, že dáta sú čisté, štruktúrované a pripravené na analýzu.
    \item \textbf{Flexibilita:} Umožňuje rôzne techniky transformácie, čím umožňuje prispôsobené obchodné logiky.
    \item \textbf{Predspracovanie:} Dáta sú transformované pred tým, ako sa dostanú do cieľového systému, čo znižuje potrebu náročných operácií v cieľovom systéme.
\end{itemize}

\section{Pochopenie ELT pipelines}

ELT (Extract, Load, Transform) je variácia, kde sa dáta najprv extrahujú a načítajú do cieľového systému, ako je dátový sklad, a potom sa transformujú pomocou výpočtovej kapacity cieľového systému. ELT získava na popularite vďaka rozmachu cloudových dátových skladov a platforiem veľkých dát, ktoré poskytujú škálovateľné výpočtové zdroje pre spracovanie veľkých dátových súborov.

Fázy ELT sú:

\begin{itemize}
    \item \textbf{Extract:} Dáta sa získavajú z rôznych zdrojov.
    \item \textbf{Load:} Surové dáta sa načítajú priamo do cieľového systému bez transformácie.
    \item \textbf{Transform:} Po načítaní sa dáta transformujú v rámci cieľového systému, zvyčajne pomocou SQL alebo iných dotazovacích jazykov.
\end{itemize}

\section{Výhody ELT pipelines}

ELT pipelines sú výhodné v situáciách, kde sa spracúvajú veľké objemy dát a cieľový systém má kapacitu na vykonanie transformácií vo veľkom rozsahu. Niektoré výhody zahŕňajú:

\begin{itemize}
    \item \textbf{Škálovateľnosť:} Moderné cloudové sklady umožňujú transformáciu veľkých dátových súborov pomocou distribuovaného výpočtového výkonu.
    \item \textbf{Jednoduchosť:} Dáta sa načítajú tak, ako sú, čo znižuje zložitosť transformácií počas fázy extrakcie.
    \item \textbf{Rýchlosť:} ELT môže byť rýchlejšie pre načítanie dát, keďže transformácie sú vykonávané až po ich načítaní.
\end{itemize}

\section{ETL vs. ELT: Ktorý prístup zvoliť?}

Výber medzi ETL a ELT závisí od konkrétneho použitia a dostupnej infraštruktúry:

\begin{itemize}
    \item \textbf{ETL:} Najvhodnejšie pre prípady, kde dáta vyžadujú významné transformácie, validáciu alebo obohatenie pred analýzou. Je ideálne pre prácu s tradičnými systémami, on-premise databázami alebo keď je obmedzená výpočtová kapacita cieľového systému.
    \item \textbf{ELT:} Preferované pri práci s modernými cloudovými dátovými platformami, ktoré dokážu spracovať veľké množstvo dát. ELT je vhodné, keď chcete rýchlo načítať surové dáta a transformácie vykonávať v cieľovom systéme.
\end{itemize}

\section{Nastavenie ETL pipeline}

Príkladový pracovný postup pre nastavenie ETL pipeline:

\begin{enumerate}
    \item \textbf{Extrakcia dát:} Použite konektory alebo API na získanie dát z rôznych zdrojových systémov, ako sú databázy, súbory alebo tretie strany.
    \item \textbf{Transformácia dát:} Spracujte dáta aplikovaním filtrov, agregácií alebo formátovania. Napríklad môžete zvýšiť všetky mzdy zamestnancov o 10 \%.
    \item \textbf{Načítanie dát:} Uložte transformované dáta do cieľového systému, ako je dátový sklad alebo dátové jazero.
\end{enumerate}

\section{Nastavenie ELT pipeline}

Na nastavenie ELT pipeline postupujte takto:

\begin{enumerate}
    \item \textbf{Extrakcia dát:} Rovnako ako pri ETL, extrahujte dáta z viacerých zdrojov.
    \item \textbf{Načítanie dát:} Načítajte surové dáta do cieľového systému, ako je cloudový dátový sklad.
    \item \textbf{Transformácia dát:} Využite výpočtovú kapacitu cieľového systému na vykonanie transformácií prostredníctvom SQL alebo iných dotazovacích jazykov.
\end{enumerate}

\section{Programatické transformácie pomocou dbt}

Dátová komunita čoraz viac využíva nástroje ako \textbf{dbt (data build tool)} na programatické transformácie v rámci pipeline. dbt umožňuje používateľom písať transformácie v SQL a poskytuje funkcie testovania a dokumentácie.

\begin{itemize}
    \item \textbf{Transformácie založené na SQL:} Používatelia môžu definovať logiku transformácií v SQL skriptoch známych ako dbt modely.
    \item \textbf{Testovanie a dokumentácia:} dbt umožňuje implementovať kontroly kvality dát a vytvárať dokumentáciu spoločne s transformáciami.
\end{itemize}

Viac informácií o dbt nájdete v \href{https://docs.getdbt.com/}{dokumentácii dbt}.

\section{Záver}

ETL a ELT pipelines hrajú zásadnú úlohu v moderných dátových pracovných tokoch tým, že zabezpečujú efektívnu integráciu, transformáciu a načítanie dát do cieľových systémov. Voľba medzi ETL a ELT závisí od faktorov, ako sú objem dát, zložitosť transformácií a dostupná infraštruktúra.

\begin{itemize}
    \item \textbf{ETL:} Ideálne pre scenáre, kde sú potrebné zložité transformácie pred načítaním dát.
    \item \textbf{ELT:} Vhodné pre moderné cloudové prostredia, ktoré dokážu spracovať veľké dátové súbory a umožňujú transformácie po načítaní.
\end{itemize}

Použitím správneho prístupu môžu organizácie zabezpečiť efektívne spracovanie dát, čo podporí lepšie rozhodovanie a analýzu.

\end{document}

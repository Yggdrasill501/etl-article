\documentclass{article}
\usepackage{graphicx}
\usepackage{hyperref}
\usepackage{listings}
\usepackage{xcolor}

\title{Sprievodca ETL a ELT Pipelineami v Pythone}
\author{}
\date{}

\lstset{language=Python,
        basicstyle=\ttfamily,
        keywordstyle=\color{blue},
        commentstyle=\color{gray},
        stringstyle=\color{orange},
        frame=single,
        breaklines=true}

\begin{document}

\maketitle

\section{Úvod}

Zabezpečenie bezproblémovej integrácie dát medzi rôznymi platformami je pre každú organizáciu kľúčové. Procesy transformácie dát sú nevyhnutné na spracovanie rôznorodých dátových súborov a moderné nástroje a techniky umožňujú efektívnejšiu správu týchto transformácií. ETL (Extract, Transform, Load) a ELT (Extract, Load, Transform) sú zásadné pre oblasť dátového inžinierstva, poskytujú rámce na presun, spracovanie a transformáciu dát.

V tomto článku sa pozrieme na ETL a ELT pipelines, ich význam a ako nastaviť tieto procesy na zjednodušenie integrácie a správy dát.

\section{Pochopenie ETL pipelines}

ETL pipelines predstavujú tradičnú metódu pre integráciu dát, kde sa dáta najprv extrahujú z rôznych zdrojových systémov, potom sa transformujú do vhodného formátu a nakoniec sa načítajú do cieľového systému, ako je dátový sklad alebo dátové jazero. ETL je obzvlášť výhodné, keď dáta potrebujú byť vyčistené, obohatené alebo preštruktúrované pred ich uložením alebo analýzou.

Tri hlavné fázy ETL sú:

\begin{itemize}
    \item \textbf{Extract:} Extrakcia dát zo zdrojových systémov, ktoré môžu zahŕňať databázy, API, súbory alebo iné zdroje dát.
    \item \textbf{Transform:} Transformácia dát, ako je filtrovanie, agregácia alebo konverzia formátov dát na splnenie obchodných požiadaviek.
    \item \textbf{Load:} Uloženie transformovaných dát do cieľového systému, zvyčajne do dátového skladu, dátového jazera alebo inej analytickej platformy.
\end{itemize}

\section{Príklad ETL pipeline v Pythone}

Tu je jednoduchý príklad ETL pipeline v Pythone, ktorý ukazuje základnú extrakciu dát, transformáciu a načítanie do súboru:

\begin{lstlisting}[caption={Základný ETL proces v Pythone}]
import pandas as pd

# Extrakcia dát z CSV súboru
data = pd.read_csv('employees.csv')

# Transformácia dát: zvýšenie mzdy o 10%
data['salary'] = data['salary'] * 1.10

# Načítanie transformovaných dát do nového CSV súboru
data.to_csv('transformed_employees.csv', index=False)

print("ETL pipeline dokončená.")
\end{lstlisting}

V tomto jednoduchom príklade extrahujeme dáta zo súboru `employees.csv`, vykonáme transformáciu tým, že zvýšime mzdy o 10\%, a výsledné dáta uložíme do nového súboru `transformed_employees.csv`.

\section{Výhody ETL pipelines}

ETL pipelines sú obzvlášť účinné, keď dáta vyžadujú komplexné transformácie alebo predspracovanie pred uložením. Hlavné výhody ETL pipelines sú:

\begin{itemize}
    \item \textbf{Kvalita dát:} Transformácia môže zabezpečiť, že dáta sú čisté, štruktúrované a pripravené na analýzu.
    \item \textbf{Flexibilita:} Umožňuje rôzne techniky transformácie, čím umožňuje prispôsobené obchodné logiky.
    \item \textbf{Predspracovanie:} Dáta sú transformované pred tým, ako sa dostanú do cieľového systému, čo znižuje potrebu náročných operácií v cieľovom systéme.
\end{itemize}

\section{Pochopenie ELT pipelines}

ELT (Extract, Load, Transform) je variácia, kde sa dáta najprv extrahujú a načítajú do cieľového systému, ako je dátový sklad, a potom sa transformujú pomocou výpočtovej kapacity cieľového systému. ELT získava na popularite vďaka rozmachu cloudových dátových skladov a platforiem veľkých dát, ktoré poskytujú škálovateľné výpočtové zdroje pre spracovanie veľkých dátových súborov.

Fázy ELT sú:

\begin{itemize}
    \item \textbf{Extract:} Dáta sa získavajú z rôznych zdrojov.
    \item \textbf{Load:} Surové dáta sa načítajú priamo do cieľového systému bez transformácie.
    \item \textbf{Transform:} Po načítaní sa dáta transformujú v rámci cieľového systému, zvyčajne pomocou SQL alebo iných dotazovacích jazykov.
\end{itemize}

\section{Výhody ELT pipelines}

ELT pipelines sú výhodné v situáciách, kde sa spracúvajú veľké objemy dát a cieľový systém má kapacitu na vykonanie transformácií vo veľkom rozsahu. Niektoré výhody zahŕňajú:

\begin{itemize}
    \item \textbf{Škálovateľnosť:} Moderné cloudové sklady umožňujú transformáciu veľkých dátových súborov pomocou distribuovaného výpočtového výkonu.
    \item \textbf{Jednoduchosť:} Dáta sa načítajú tak, ako sú, čo znižuje zložitosť transformácií počas fázy extrakcie.
    \item \textbf{Rýchlosť:} ELT môže byť rýchlejšie pre načítanie dát, keďže transformácie sú vykonávané až po ich načítaní.
\end{itemize}

\section{Záver}

ETL a ELT pipelines hrajú zásadnú úlohu v moderných dátových pracovných tokoch tým, že zabezpečujú efektívnu integráciu, transformáciu a načítanie dát do cieľových systémov. Voľba medzi ETL a ELT závisí od faktorov, ako sú objem dát, zložitosť transformácií a dostupná infraštruktúra.

Použitím správneho prístupu môžu organizácie zabezpečiť efektívne spracovanie dát, čo podporí lepšie rozhodovanie a analýzu.

\end{document}
